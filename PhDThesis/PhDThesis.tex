%% sample template file for a PhD Thesis
%%
%% Choose format: either a4 or b5 by uncommenting the respective line
\documentclass[11pt,openright]{book} %% use when print out will not be further scaled
% \documentclass[12pt,openright]{book} %% use instead when print-out is scaled down to B5

\usepackage[a4paper]{geometry} %% use wider a4paper definition from geometry package
% \usepackage[b5paper]{geometry} %% use instead when print-out is in B5

%% Character input setup, choose one of the following
%% to be able to use western european special chars directly
\usepackage[utf8]{inputenc}  %% for modern UTF-8 OSes (Linux,Win7)
% \usepackage[latin1]{inputenc}  %% for outdated latin1 based systems (WinXP)

\usepackage{graphicx}         %% needed for inclusion of graphics
\graphicspath{{figs/}}         %% this is were the images reside

\usepackage[british]{babel}   %% some british specific settings

\usepackage[T1]{fontenc}      %% recommended for readable pdf
\usepackage{lmodern}          %% use vector fonts
\usepackage{amsmath,mathptmx} %% we need lots of math features
\usepackage{url}              %% helps displaying URLs
\usepackage[pdfborder={0 0 0}]{hyperref} %% generate links in the PDF (but no boxes around them)
\usepackage{longtable}        %% for tables then can go over several pages

%% Bibliography setup %%
%%% default is "number-only" style
\bibliographystyle{plain}
%%% If you want to use "author-date" style
%%% where `\citet{Foo2011}` generates "Foo et al. (2011)"
%%% and   `\citep{Foo2011}` generates "(Foo et al. 2011)"
%%% then comment the line above and use
% \usepackage{natbib}
%%% or
%%% if you want to use "author-number" style then use
% \usepackage[numbers]{natbib}
%%% instead.

\usepackage{lipsum} % just for providing fill text used in this template

\begin{document}
\frontmatter  %% switches to roman page numbers
\begin{titlepage}
  \vspace*{\stretch{1}}
  % Title
  \begin{center}
    \Huge{The Title of the PhD thesis}
  \end{center}
  \vspace*{\stretch{0.5}}
  % Author
  \begin{center}
    \Large{John Smith}
  \end{center}
  \vspace*{\stretch{0.5}}
  % Submission statement (DO NOT CHANGE)
  \begin{center}
    \large{Thesis submitted to the Telemark University College\\ for the degree of philosophiae doctor~(PhD)}
  \end{center}
  \vspace*{\stretch{2}}
\end{titlepage}\cleardoublepage
% \maketitle\thispagestyle{empty} %% We don't want a page number on the title page
\pdfbookmark[0]{Titlepage}{title} % Sets a PDF bookmark for the title page

% Dedication page if wished:
\vspace*{\stretch{1}}
\pdfbookmark[0]{Dedication}{dedication} % Sets a PDF bookmark for the dedication
\begin{flushright}
  \large{\emph{Dedicated to the teachers of TUC ;-)}}

\end{flushright}
\vspace*{\stretch{2}}



\chapter*{Preface}
\label{sec:preface}
\addcontentsline{toc}{chapter}{Preface}
\lipsum[1-3]

\chapter*{Summary}
\label{sec:summary}
\addcontentsline{toc}{chapter}{Summary}
\lipsum[1-3]


\tableofcontents
\addcontentsline{toc}{chapter}{Contents}
\listoffigures % out-comment when not wished
\addcontentsline{toc}{section}{List of Figures}

\listoftables  % out-comment when not wished
\addcontentsline{toc}{section}{List of Tables}

\chapter*{Nomenclature}
\label{sec:summary}
bla

\begin{longtable}{ll}
  \textbf{Symbol} & \textbf{Explanation}\endhead\\
  A/D	& Analogue-Digital-Converter \\
  CMR	& Common Mode Rejection \\
  foo	& Foo \\
  bar 	& Bar
\end{longtable}


\mainmatter
\part{Overview}  %% DO NOT USE \part FOR MONOGRAPHY
\label{part:overview}
\chapter{Introduction}
\label{ch:intro}
\lipsum[4]
\begin{figure}[!ht]
  \centering
  \includegraphics[width=0.8\textwidth]{HiT-logo}
  \caption{The HiT logo in colour}
  \label{fig:hit-logo}
\end{figure}
\lipsum[4]
\begin{figure}[!ht]
  \centering
  \includegraphics[width=0.8\textwidth]{HiT-logo_bw}
  \caption{The HiT logo in black and white}
  \label{fig:hit-logo-bw}
\end{figure}
\lipsum

\section{Background}
\label{sec:back}
\lipsum[4]
\begin{equation}
  e = m c^2
\end{equation}
\lipsum

\chapter{Theory}
\label{ch:theory}

\section{Maxwell's Equations}
\label{sec:theory}
\indent The differential forms of Maxwell's equations as found by Heaviside, while completely valid, are now considered somewhat archaic, and have been replaced by the more useful (equivalent) integral forms. Each law is named according to the person(s) who originally discovered the connections represented by the equation. Here are the four equations:
\begin{eqnarray}
  \text{Gauss' law for electricity:}& \displaystyle \oint{\vec{E}\cdot\mathrm{d}\vec{A}}&=\frac{Q_{enc}}{\epsilon_0}\\
  \text{Gauss' law for magnetism:}& \displaystyle \oint{\vec{B}\cdot\mathrm{d}\vec{A}}&=0\\
  \text{Faraday's law:}& \displaystyle\oint{\vec{E}\cdot\mathrm{d}\vec{s}}&=-\frac{\emph{d}\phi_b}{\mathrm{d}t}\\
  \text{Ampere-Maxwell law:}& \displaystyle\oint{\vec{B}\cdot\mathrm{d}\vec{s}}&=\mu_0\epsilon_0\frac{\emph{d}\phi_e}{\mathrm{d}t}+\mu_0 i_{enc}
\end{eqnarray}
Note: $\oint$ is used to specify a closed loop integral, also known as a line integral. It simply means that in the calculations, we must go all the way around the loop; we can't stop part way through or the equations won't be valid.

\section{Mathematical model}
\label{sec:back}
\lipsum[8]
\begin{table}[!ht]
  \caption{The different number systems}
  \centering
  \begin{tabular}{|r|l|}
    \hline
    7C0 & hexadecimal \\
    3700 & octal \\ \cline{2-2}
    11111000000 & binary \\
    \hline \hline
    1984 & decimal \\
    \hline
  \end{tabular}
\end{table}

\lipsum[4]

\begin{table}[!ht]
 \caption{The weather forecast}
  \centering
   \begin{tabular}{ | l | l | l | p{5cm} |}
    \hline
    Day & Min Temp & Max Temp & Summary \\ \hline
    Monday & 11C & 22C & A clear day with lots of sunshine.
    However, the strong breeze will bring down the temperatures. \\ \hline
    Tuesday & 9C & 19C & Cloudy with rain, across many northern regions. Clear spells
    across most of Scotland and Northern Ireland,
    but rain reaching the far northwest. \\ \hline
    Wednesday & 10C & 21C & Rain will still linger for the morning.
    Conditions will improve by early afternoon and continue
    throughout the evening. \\
    \hline
    \end{tabular}
\end{table}
\lipsum{100-150}

~\nocite{*}

\cleardoublepage
\phantomsection
\bibliography{vancouver.bib}
\addcontentsline{toc}{chapter}{Bibliography}


\appendix
\renewcommand{\appendixname}{Paper} %% So we get 'Paper X' displayed instead

\part{Published and Submitted Papers}  %% DO NOT USE \part FOR MONOGRAPHY
\label{part:papers}


\chapter{Title of Paper A}
\label{paper-a}

Short descriptive text of paper follows here.

The paper itself will needs to be included in the published form as pdf on the next pages.

\chapter{Title of Paper B}
\label{paper-b}
Short descriptive text of paper follows here.

The paper itself will needs to be included in the published form as pdf on the next pages.

\end{document}
